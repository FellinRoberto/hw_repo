\documentclass[12pt]{report}

\linespread{1.3}

\title{Homework 5}
\author{Chang Wang}

\begin{document}

\maketitle

\section{Restriction}
\subsection{LONGEST PATH (LP)}
In this case, we can reduce Hamiltonian Circle Problem (HC) to LP. \\
Given a graph $G=(V, E)$, if there is a HC in $G$, it is clearly that the path in HC is the longest path in $G$. Because HC covers all the vertices in $G$ and length of path is $|V|=k$. \\
Conversely, if there is a LP in $G$, which tells us the length of the path is $k=|V|$. So the longest path visits all the vertices in $G$ without repetition. This is the definition of HC. \\
So it is clearly that $LP \in NP-C$.

\subsection{SET PACKING (SP)}
In this case, we use 3DM problem to reduce to SP. Actually, 3DM is a special case of SP.\\
Suppose each element in 3DM, $d_{i} = \{w_{f(i)}, x_{g(i)}, y_{h(i)}\} \in W \cup X \cup Y$ and $d_{i} \cap d_{j} = \phi, 1 \le i, j \le |W|=|X|=|Y|$, this satisfies the definition of SP. \\
Conversely, suppose there is a SP, and each subset $s_{i} \in S$ we have $|s_{i}| = 3$ and $s_{i} \cap s_{j} = \phi$. Then from each $s_{i}$ pick up one element into W, one element in X and one element into Y, then the constructed structure is definitely a 3DM. \\
$SP \in NP-C$ is proven.

\subsection{PARTITION INTO HAMILTONIAN SUBGRAPHS (PHS)}
Apparently, PHS is similar to Partition into Triangles problem (PIT), if for each $V_{k}$ where $|V_{k}| = 3$ and $\sum_{i=1}^{k}|V_{i}| = |V|$, of course triangle is a Hamiltonian Circuit, PHS could be considered as PIT problem, so actually PIT problem is a just a special case of PHS problem. \\
Given a graph $G = (V, E)$, $|V| = 3q$ for a positive integer q. if there is a partition of $V$ into $q$ disjoint sets $V_{1}, V_{2}, ..., V_{q}$, we can say that $G$ is partitioned into $q$ subgraphs which have a Hamiltonian Circuits. \\
Then the problem is proven that $PHS \in NP-C$.

\section{Local Replacement}

\subsection{FEEDBACK VERTEX SET (FVS)}
This problem looks pretty like Vertex Cover Problem, so I'd like to transform VC to FVS. \\
First given a graph G=(V,E), $K \le |V|$, and $G$ is a undirected graph. We reduce $G^{'}$ from G, where $G^{'} = (V^{'}, E^{'})$ is a directed graph and $G$ has a vertex cover of size K iff $G^{'}$ has a feedback vertex set of size $K^{'}$. \\
We construct $G^{'}$ is to have coverage of directed cycles in $G^{'}$ correspond to coverage of edges in G. Replacing each edge $(u,v) \in E$ by two directed edges $(u^{'}, v^{'}), (v^{'}, u^{'}) \in E^{'}$. So now $V^{'} = V$ and $E^{'} = 2E$. Here take a look at the transformation function, substitute one edge in $G$ to two edges in $G^{'}$. The edge is related to the vertex in G, the worst case is $|E| = \frac {|V| \times (|V|-1)}{2}$, so it takes $\frac {|V| \times (|V|-1)}{2}$ step to finish the transformation at most. Apparently it is polynomial transformation. \\
If there is a vertex cover $V_{vc}$ in G, then it covers every directed cycle in $G^{'}$, because any cycle in $G^{'}$ involves a edge in G. \\
Conversely, if there is a Feedback Vertex Set $V_{fvs}$ covers all the directed cycles in $G^{'}$, we know that each edge in $G^{'}$ correspond to a edge in G, which means $V_{fvs}$ covers all the edge in G. \\
Then the problem is proven that $FVS \in NP-C$.

\subsection{EXACT COVER BY 4-SETS (X4C)}
Of course, to prove this problem, the simplest choice is Exact Cover by 3-Sets (X3C). \\
Given a finite set X with $|X|=3q$ and a collection C of 3-element subsets of X. Then we reduce this structure to X4C problem. Assume each element $c_{j} \in C$ where $1 \le j \le q$ and $|c_{j}| = 3$. For each element $c_{j}$, add a distinct element $k_{j}$ where $1 \le j \le q$. Now we have a finite set $X^{'} = X \cup (\bigcup_{j=1}^{q}k_{j})$, where $|X^{'}| = 4q$, and a collection $C^{'}$ of 4-element subsets of X, where $c_{j}^{'} = c_{j} \cup k_{j}$. It is easy to see that this transformation takes q steps to complete, which related to the size of collection C. Then it is a polynomial transformation. \\
If the set X contains a 3-element cover, due to the elements $k_{j}$ added is distinct from each other and existing elements, it is clearly that for any subset in $C^{'}$ of $X^{'}$, that $c^{'}_{i}, c^{'}_{j} \in C^{'}$, we have $c^{'}_{i} \cap c^{'}_{j} = \phi$, which satisfies the definition of X4C. \\
If the set $X^{'}$ contains a 4-element cover, for any subset $c^{'}_{i}, c^{'}_{j} \in C^{'}$, there is no common element in any $c^{'}_{i}$ and $c^{'}_{j}$, so we can remove any single element in each $c^{'}_{j}$, $|X| = |X^{'}| - q = 3q$, and they will remain the same configuration that $c^{'}_{i} \cap c^{'}_{j} = \phi$, and this is the definition of X3C. \\
Now the problem is proven that $X4C \in NP-C$.

\subsection{DOMINATING SET (DS)}
According to the definition of DS, it is similar to Vertex Cover problem. One is required to dominate all the edges, one is required to dominate all the vertex. \\
Now we build a new vertex $v_{e}$ for each edge $e \in G$, then each edge $e = (u, v)$ of $G$ is replaced by a triangle of edges: $(u, v), (u, v_{e}), (v, v_{e})$. The new graph $G^{'} = (V^{'}, E^{'})$, where $V^{'} = V \cup \bigcup^{|V|}v_{e}$ and $E^{'} = E \cup \bigcup_{v \in E}(v, v_{e})$. The transformation is to create new vertex and edges, which is related to the size of vertex, so it is a polynomial transformation. \\
Now we have to prove $G$ has a vertex cover of size K iff $G^{'}$ has dominating set of size K. If there is a vertex cover $V_{vc}$ in G, it is true there is a Dominating Set $V_{vc}$ in $G^{'}$. Because every edge of $G$ has a vertex in $V_{vc}$ incident to it, and so every triangle of vertices in $G^{'}$ has at least one member in $V_{vc}$, and so every vertex of $G^{'}$ is either in $V_{vc}$ or $V - V_{vc}$. If $G$ has a vertex cover K, then $G^{'}$ has a dominating set of size K. \\
Then if there a Dominating Set $V_{ds}$ in $G^{'}$. Firstly, if all vertices in $V_{ds}$ belong to G, it is apparently that $V_{ds}$ is also a Vertex Cover in G. Because according to the definition of DS, there is always a edge connected vertices from $V_{ds}$ to $V^{'} - V_{ds}$. This is a Vertex Cover. Another situation is, if there are some vertices in $V_{ds}$ is the created edge $v_{e}$ rather than a vertex in G, then we have to replace it by one of the vertex in $G$ for the edge's endpoints and it will still be dominating set. Because $v_{e}$ only dominate 3 vertices including itself in the triangle. The other two vertices in this triangle also at least dominate 3 vertices. Once finished replacing all the vertices $v_{e}$ in $V_{ds}$, the new $V_{ds}$ forms a vertex cover of G, because every edge of $G$ must have at least one of its endpoints in $V_{ds}$ for $V_{ds}$ to be a dominating set of $G^{'}$. \\
we finished proving that $DS \in NP-C$.







\end{document}