\documentclass[12pt,a4paper]{report}
\usepackage{amssymb,amsmath}

\title{Homework 4}
\author{Chang Wang}
\begin{document}

\maketitle

\section{It is required to compute the sum of 32K 32-bit numbers. Estimate the time required for this computation on:}
\subsection{A 32-bit SISD}
Before getting into the detail, here are some assumptions. \\[0.15cm]
1. The ADD operation takes $T_{1}$ cycles; \\[0.15cm]
2. The communication between processors takes $T_{2}$ cycles. \\[0.15cm]
In this case, there is only one processor to compute the numbers, it is a serial operation.
\begin{equation*}
\begin{split}
Time_{1}& = (32 * 2^{10} - 1) * T_{1} \\
		& \approx 2^{15} * T_{1}
\end{split}
\end{equation*}

\subsection{A 16-bit SISD}
In this case, there still is one processor, which but is 16-bit. In order to compute 32-bit numbers, here we assume it computes low 16-bit first, then compute high 16-bit and carry on point if it has. So this is also a serial operation.
\begin{equation*}
\begin{split}
Time_{2}& = (32 * 2^{10} - 1) * 2T_{1} \\
         & \approx 2^{16} * T_{1}
\end{split}
\end{equation*}

\subsection{An SIMD with 32 32-bit processors}
In this case, we have 32 arithmetic-logic processors which can work simultaneously. At first, each processor computes $2^{10}$ numbers at the same time, which are serial operations; then there are 32 partial sums. Now pair each two processors together to compute the sum of whole numbers.
\begin{equation*}
\begin{split}
Time_{3}& = (2^{10} - 1) * T_{1} + \log_{2} 32 * (T_{1} + T_{2}) \\
& \approx 2^{10} * T_{1} + 5 * (T_{1} + T_{2})
\end{split}
\end{equation*}

If $T_{2}$ is approaching $T_{1}$, we could omit $T_{2}$ here. But in practice, communication between processors definitely is a overloaded work, which can not easily be ignored.
\subsection{An SIMD with 16 32-bit processors}
In this case, using the same approach above. \\
\begin{equation*}
\begin{split}
Time_{4}& = (2^{10} -1 ) * 2T_{1} + \log_{2}32 * 2(T_{1} + T_{2}) \\
& \approx 2^{11} * T_{1} + 10 * (T_{1} + T_{2})
\end{split}
\end{equation*}

\section{What is the best way to implement the computation of Problem (2.8) on an MIMD? That is, how do you minimize the overhead involved?}
If it just needs to minimize the overhead, we can use only one arithmetic-logic processor, then there is no communication cost involved. But it maximizes the computing cost at the same time. So we should find a balance to minimize the whole cost.
Suppose there are n 32-bit processors (without losing generality, assume n is a factor of $32 * 2^{10}$.) At the beginning, n processors compute simultaneously, after that there will be n partial sums. Then each two processors communicate each other to compute the sum.
\begin{equation*}
\begin{split}
Time & = (\frac {32 * 2^{10}} {n} - 1) * T_{1} + \log_{2}n * (T_{1} + T_{2}) \\
& \approx (\frac {32 * 2^{10}} {n}) * T_{1} + \log_{2}n * (T_{1} + T_{2})
\end{split}
\end{equation*}
If we know the relation between $T_{1}$ and $T_{2}$, then we can solve the equation to get its minimum value.
\end{document}